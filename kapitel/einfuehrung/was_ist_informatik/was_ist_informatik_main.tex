% !TEX root = ../../../main.tex

\toggletrue{image}
\togglefalse{imagehover}
\chapterimage{purity}
\chapterimagetitle{\uppercase{Purity}}
\chapterimageurl{https://xkcd.com/435/}

\chapter{Was ist Informatik?}
\label{chapter-was-ist-informatik}

\section{Eine Definition}

Es gibt verschiedene Definitionen, die versuchen die Informatik zu erklären. Prof. Dr. Walter Gander und Prof. Dr. Juraj Hromkovič (Ausbildungszentrum und Beratungszentrum für Informatikunterricht) versuchen
dies wie folgt:

\begin{definition}[Informatik]
Die Informatik ist die Wissenschaft der systematischen, \textbf{automatisierten Verarbeitung} von Information, der Informationsspeicherung, Informationsverwaltung und Informationsübertragung \cite{def-informatik}.
\end{definition}

Das Ziel des Grundlagenfachs ist Ihnen die Informatik als \textbf{Wissenschaft} vorzustellen. Wir werden uns \textbf{nicht} darum kümmern, wie man Microsoft Word bedient, im Internet surft oder wie man Spiele spielt. Es geht jedoch zum Beispiel darum, zu verstehen wie ein Computer Text speichern kann, wie die Kommunikation über das Internet funktioniert oder wie man ein kleines Spiel programmieren kann. Sie sollen vom Konsumenten zum Produzenten von Informatikinhalten werden! Im englischen Sprachraum wird für die Informatik typischerweise der Begriff \textbf{Computer Science} (\say{Computerwissenschaft}) verwendet.

\section{Was machen Informatiker?}

Die Informatik ist eine interdisziplinäre Wissenschaft. Informatiker arbeiten häufig mit anderen Fachgebieten zusammen. Einige \say{Partnerwissenschaften} und Inhalte die sich beide Wissenschaften teilen.

\begin{itemize}
\item Metawissenschaften (z.B. Mathematik): Was ist Zufall? Wie kann man die Korrektheit eines Programms beweisen?
\item Sozialwissenschaften: Welchen Einfluss hat die künstliche Intelligenz auf unser Leben?
\item Naturwissenschaften: Wie kann eine Artbestimmung (z.B. Fledermausarten) erfolgen?
\item Ingenieurwissenschaften: Wie kann man den Gütertransport auf der Schiene überwachen?
\end{itemize}

\section{Wie wird man Informatiker?}

\begin{itemize}
\item Berufslehre: Applikationsentwicklung, Systemtechnik, Betriebsinformatik
\item Studium: Informatik, Wirtschaftsinformatik, Bioinformatik, ...
\end{itemize}

