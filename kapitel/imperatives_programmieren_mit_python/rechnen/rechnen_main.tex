% !TEX root = ../../../main.tex

\toggletrue{image}
\toggletrue{imagehover}
\chapterimage{wrong_times_table}
\chapterimagetitle{\uppercase{Wrong Times Table}}
\chapterimageurl{https://xkcd.com/2313/}
\chapterimagehover{Deep in some corner of my heart, I suspect that real times tables are wrong about 6x7=42 and 8x7=56.}

\chapter{Rechnen}
\label{chapter-rechnen}

Mit einer Programmiersprache können wir natürlich auch Berechnungen durchführen. Dies war schliesslich auch die ursprüngliche Motivation für die Konstruktion von Computern\footnote{\say{to compute} kann mit rechnen oder etwas berechnen übersetzt werden. Im deutschen Sprachraum ist auch Rechner als Bezeichnung für den Computer üblich.}. In diesem Kapitel beschäftigen wir uns hauptsächlich mit den grundlegenden Rechenarten, welche wir alle aus dem Mathematikunterricht kennen. Die Lernziele lauten:

\newcommand{\rechnenLernziele}{
\begin{todolist}
\item Sie erstellen Python-Programme, in denen Sie mit Zahlen rechnen.
\item Sie erklären, was ein arithmetischer Operator ist und geben ein Beispiel.
\item Sie erklären, was ein arithmetischer Ausdruck ist und geben ein Beispiel.
\end{todolist}
}

\lernziel{\autoref{chapter-rechnen}, \nameref{chapter-rechnen}}{\protect\rechnenLernziele}

\rechnenLernziele

\section{Addition, Subtraktion, Multiplikation, Division und Potenzierung}

Wie beim Taschenrechner gibt es für jede Rechenart ein Zeichen. 

\begin{multicols}{2}
\begin{itemize}
	\item Addition: \lstinline{+}
	\item Subtraktion: \lstinline{-}
	\item Multiplikation: \lstinline{*}
	\item Division: \lstinline{/}
	\item ganzzahlige Division: \lstinline{//}
	\item Potenzierung: \lstinline{**}
\end{itemize}
\end{multicols}

\begin{definition}[Arithmetischer Operator]
Der Fachbegriff für \say{Rechenzeichen} lautet in der Programmierung arithmetischer Operator. Mit arithmetischen Operatoren notieren wir mathematische Rechnungen.
\end{definition}

Die Arithmetik ist ein Teilgebiet der Mathematik und beschreibt das Rechnen mit Zahlen.

\begin{example}
\autoref{lst-calc-print-1} zeigt Beispiele in Kombination mit \lstinline{print}-Funktionsaufrufen.

\begin{figure}[htb]
\centering
\begin{minipage}{0.6\textwidth}
\centering
\begin{lstlisting}[label={lst-calc-print-1}, caption={\graybgtexttt{rechenbeispiel\_1.py}}]
print(5 + 3)
print(<@\color{textcolor}f@>"Ergebnis: <@\color{keywordcolor}\{\color{black}5 - 3\color{keywordcolor}\}@>")
print(<@\color{textcolor}f@>"Erst Punkt: <@\color{keywordcolor}\{\color{black}3.7 + 5 * 3\color{keywordcolor}\}@>")
print(5 / 3)
print(2 ** 16)
\end{lstlisting}
\end{minipage}
\hfill
\begin{minipage}{0.35\textwidth}
\centering
\begin{lstlisting}[language=output, caption={Konsolenausgabe}, label={lst-calc-print-output-1}]
8
Ergebnis: 2
Punkt zuerst: 18.7
1.6666666666666667
65536
\end{lstlisting}
\end{minipage}
\end{figure}

In der Konsole (siehe \autoref{lst-calc-print-output-1}) ist zu sehen, dass Zahlen nicht immer exakt berechnet werden. Ausserdem kann man erkennen, dass Python die \say{Punkt-vor-Strich-Regel} beachtet (Rechnung in Zeile \num{3}). Formatierter Text ist auch in der Lage, eine Rechnung zu beinhalten. Die Rechnung wird automatisch ausgerechnet.

\end{example}

\begin{important}
Möchte man \say{Kommazahlen} in Rechnungen verwenden, so muss man einen Punkt für das Komma verwenden!
\end{important}

\subsection{Ganzzahlige Division, Klammern und Variablen}

Die Division mit dem doppelten Schrägstrich (eng. double forward slash) bewirkt, dass nur der \textbf{ganzzahlige Anteil} der Division als Ergebnis herauskommt. Es wird also die gewöhnliche Division berechnet und dann zur \textbf{nächsten ganzen Zahl abgerundet} (mathematische Abrundungsfunktion) benutzt. Möchte man die Reihenfolge der Berechnungen beeinflussen, dann kann man wie in der Mathematik \textbf{runde} Klammern verwenden. Natürlich kann man bei Rechnungen auch Variablen und Werte kombinieren.

\begin{figure}[htb]
\centering
\begin{minipage}{0.5\textwidth}
\centering
\begin{lstlisting}
import random as r

print(7 // 2)
print(1 // 2)
print(9.0 // 3)
a = r.randrange(1, 11)
b = r.randrange(1, 11)
ergebnis = (a + b) * (a - b)
\end{lstlisting}
\end{minipage}
\hfill
\begin{minipage}{0.45\textwidth}
\centering
\begin{lstlisting}
import random as r

a = r.randrange(1, 101)
print(<@\color{textcolor}f@>"Seitenlänge: <@\color{keywordcolor}\{\color{black}a\color{keywordcolor}\}@>")
umfang = a + a + a + a
flaeche = a ** 2
print(<@\color{textcolor}f@>"Umfang: <@\color{keywordcolor}\{\color{black}umfang\color{keywordcolor}\}@>")
print(<@\color{textcolor}f@>"Fläche: <@\color{keywordcolor}\{\color{black}flaeche\color{keywordcolor}\}@>")\end{lstlisting}
\end{minipage}
\caption{Weitere Beispiele wie man in Python rechnen kann (\graybgtexttt{rechenbeispiel\_2.py}). Bei einer Zuweisung wird immer \textbf{zuerst} die \textbf{rechte Seite} des Zuweisungsoperators ausgewertet. Das Ergebnis wird dann in der Variablen auf der \textbf{linken Seite} des Zuweisungsoperators gespeichert.}
\label{lst-calc-print-2}
\end{figure}

\begin{important}
	Im Gegensatz zur Mathematik müssen immer alle \textbf{arithmetischen Operatoren} notiert werden. Ein fehlender arithmetischer Operator, wie in folgendem Code,
	
\begin{lstlisting}
import random as r
		
x = r.randrange(1, 11)
ergebnis = 4x
\end{lstlisting}

erzeugt einen Fehler.

\end{important}

\subsection{Clean Code}

Eine weitere Regel vereinheitlicht die Darstellung einer Rechnung.

\begin{cleancode}[Leerzeichen 3]
Links und rechts eines arithmetischen Operators fügen wir je ein Leerzeichen ein.
\end{cleancode}

\begin{example}
Wir notieren die Addition von \num{5} und \num{3} somit wie folgt: \lstinline{5 + 3}. Falsch, im Sinne von Clean Code, wäre \lstinline{5+3}.
\end{example}

\newpage

\subsection{Arithmetischer Ausdruck}

Mathematische Rechnungen sind Programmierbefehle, die in Python eine eigene Kategorie darstellen.

\begin{definition}[Arithmetischer Ausdruck]
	Rechnungen mit \textbf{arithmetischen Operatoren} werden arithmetische Ausdrücke (eng. arithmetic expressions) genannt. Bei der Programmausführung werden arithmetische Ausdrücke stets von Python direkt \textbf{ausgewertet}. Dies bedeutet, Python ermittelt für den arithmetischen Ausdruck einen \textbf{Wert} (\say{das Ergebnis der Rechnung}). Bei arithmetischen Ausdrücken ist der Wert immer eine Zahl (entweder eine ganze Zahl oder eine Fliesskommazahl). 
\end{definition}

\begin{example}
Schauen wir uns \lstinline{(163 * 3) - (77  * 4)} im Detail an:
$$
\begin{array}[t]{c}
\begin{array}[t]{cccccccccccc} 
\underbrace{\textrm{\texttt{(}}}_{\textrm{Klammer}} & \underbrace{\textrm{\texttt{163}}}_{\textrm{Wert}} & \underbrace{\textrm{\texttt{*}}}_{\textrm{arith. Op.}} & \underbrace{\textrm{\texttt{3}}}_{\textrm{Wert}} & \underbrace{\textrm{\texttt{)}}}_{\textrm{Klammer}} & \underbrace{\textrm{\texttt{-}}}_{\textrm{arith. Op.}} & \underbrace{\textrm{\texttt{(}}}_{\textrm{Klammer}} & \underbrace{\textrm{\texttt{77}}}_{\textrm{Wert}} & \underbrace{\textrm{\texttt{*}}}_{\textrm{arith. Op.}} & \underbrace{\textrm{\texttt{4}}}_{\textrm{Wert}} & \underbrace{\textrm{\texttt{)}}}_{\textrm{Klammer}}
\end{array} \\
\underbrace{\hspace{14cm}}_{\textrm{arithmetischer Ausdruck}}
\end{array}
$$
Mit arith. Op. ist der Begriff arithmetischer Operator gemeint. Würden wir den arithmetischen Ausdruck mit einem \lstinline{print}-Funktionsaufruf kombinieren, dann würde Python in der Konsole die Zahl \num{181} ausgeben.

\begin{lstlisting}
print((163 * 3) - (77  * 4))
\end{lstlisting}

\end{example}

\section{Aufgaben}

\subsection{Aufgabe 1}

Würfel (Volumen $a^3$, Oberflächeninhalt $6 \cdot a^2$)

\subsection{Aufgabe 2}

Turtle Siebeneck mit Division und Variable, for-Schleife, range

\subsection{Aufgabe 3}

Drei zufällige Zahlen, arithmetisches Mittel (Durchschnitt) berechnen und ausgeben

